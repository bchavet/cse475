\documentclass{article}

\title{CSE475 Gameday Strategy}
\author{Team Win: Ben Chavet, Justin McKinstry, Steve Mott}
\date{\today}

\begin{document}
\maketitle

Since we have three team members, we will split our strategy into three rolls:
Bidder, Money Tracker, and Utility Tracker.

The Bidder will use the information given to him by the other two team members
and the auction protocol being used to decide what the bid should be.

The Money Tracker will keep a tally of how much money each of the other teams
have left in order to help the Bidder decide what the best bid is.

The Utility Tracker will speculate about the other teams' utilities based on
their bidding trends. He will watch for potential "Yahtzee" utilities and keep
track of how many each team has. This information is given to the Bidder so the
bidder can speculate about the competition for the bid.

Our overall strategy will follow with the dominant strategies described in
class for the different auction protocols.  Each strategy, however, will be altered to accommodate for the fact that other teams will most likely be basing their strategies in a similar manner.

General Rules:
All bids under the price of the item will be invalidated by the auctioneer, so we will not be doing ANY bids under the value of the item, if given.
If we have any items that have an extremely low value, we will not even enter into the bidding.  Entering bidding costs money, and is better spent in other venues.
Using percentages, say 40\% of value for items we don't' need at all, 80\% for items we could use but don't need, 100\% for items that we wouldn't mind having, and 125\% for items we absolutely must have.

\begin{description}
  \item \textbf{English:}
    Bid 10\%/20\% higher than last bid until bidding would surpass our given value.
  \item \textbf{English open exit:}
    Same as English, but exit if bidding would put us over our given value.
  \item \textbf{Dutch:}
    Bid at the first value that is at or below our decided value.
  \item \textbf{First price sealed bid:}
    Bid our given value.
  \item \textbf{Vickrey:}
    Bid our given value.
\end{description}

With our extra information about the money and utilities of the other teams, we
can modify the strategies, though. If we are fairly confident that the teams
with high utilities for specific items are low on money, we can bid just above
their limit to ensure they don't gain utility while we gain a little bit. This
strategy only works if we monitor our money very closely too, though. If we
concentrate too hard on exploiting other teams, we will waste our money and
end up not winning our high utility items.

The reason we don't just follow the dominant strategies to the letter is
because we are assuming that the gameday is set up in a fashion that if all
teams follow the strategy to the letter, it will end in a tie. This assumption
may have to be modified based on our observations of the first gameday. We are
also assuming that with our extra person and greater capacity for information
gathering, we will have an advantage over the other teams and be able to make
better informed decisions quicker.

\end{document}