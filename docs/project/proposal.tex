\documentclass{article}

\title{CSE475 Project Proposal}
\author{Ben Chavet, Justin McKinstry, Steve Mott}
\date{\today}

\begin{document}
\maketitle

\section{Problem Statement}

The objective of this project is to design a multiagent system where
individual agents work together to effectively and efficiently extinguish
a fire in a specific area.  The area's size is $L \times L$ square units,
each with its own ``flammability'' value, where the entire area has a total
flammability value of $F$.  There are $K$ fires initially
started in random locations, $N$ firefighters randomly dropped into
the area.  Each firefighter has an initial ``energy'' level $E$, that
determines their effectiveness, as well as their life.

The fire is able to spread if not attended to quickly enough, and is
also capable of killing firefighters.  If a firefighter is killed, any
firefighters in a directly adjacent location becomes less effective at
fighting the fire.

A fire can be extinguished in two ways.  It can burn itself out, which is a
function of time and the flammability of the space occupied.  Or, it can be
extinguished by one or more firefighters.  Either way, a fire is considered
extinguished when its strength is reduced to zero.  As a fire burns, its
intensity increases based on the flammability of the space it occupies.
Once the flammability is sufficiently low, the intensity of the fire begins
to decline due to the reduced amount of fuel it has to burn.

The simulation is ended when there are no remaining fires.  If all of the
firefighters in the area are killed, the fire is allowed to burn itself out,
even if there are still firefighters in reserve.  In order to be optimally
efficient, the fire must be completely extinguished by the firefighters,
and all of the firefighters must survive.

\section{Agent Design Strategy}

There are two types of agents, fire agents and firefighter agents.  Firefighter
agents are free to move about the environment one square at a time in any
direction (north, south, east, west).  Fire agents are not allowed to move, but
can spawn new fire agents in adjacent spaces if the conditions are right.  The
goal of the firefighter agents is to extinguish the fire agents.  This is
accomplished by moving to a space adjacent to a fire and spraying water in
the fire's direction.

Firefighter agents make their decisions based on the following two goals,
in this order:

\begin{enumerate}
  \item Stay alive (keep their own energy level above zero)
  \item extinguish the fire (achieved by spraying a fire)
  \item maximize their own energy level (achieved by spraying a fire)
\end{enumerate}

In order to stay alive, a firefighter agent must keep its energy level above
zero.  If a firefighter's energy level reaches zero, that firefighter dies.
When a firefighter dies, the energy level of any adjacent firefighters is
reduced by a percentage, $P$, to simulate the emotional stress of losing a
friend.

As a firefighter sprays a fire, the intensity of the fire is reduced, and the
firefighter's energy increases.  This is how the system rewards a firefighter
agent, and motivates a firefighter to attempt to extinguish a fire.

If a firefighter and a fire occupy the same space, the firefighter loses
energy points.  This is how the system applies a cost to a firefighter's
actions.  A firefighter must observe the flammability of the space it occupies
as well as the intensity of any adjacent fires to determine the risk of the
fire spreading vs the cost that it would incur if the fire did spread.  This
information is used by the firefighter agent in order to decide whether
spraying the fire or moving to another location would maximize its utility.
This utility is:

\[ U_{action} = Reward_{action} - Risk_{action} \]

Firefighter agents are able to communicate with all other firefighter agents
in the same manner that real firefighters communicate with each other using
two way radios.  A firefighter broadcasts to the other agents when they have
located a fire, or when they have extinguished a fire and there are no adjacent
fires.  In the latter case, the firefighter would wait a sufficient amount of
time for a response and move toward the firefighter that needs the most help.

A firefighter agent bases all of its decisions solely on its own observations.

\section{Desired Emergent Behavior}

The desired emergent behavior is that the firefighters are able to
successfully extinguish all of the fire in the environment, preserving
the lives of all of the firefighters, and maintaining the highest
amount of collective energy.

The flammability of a space can be modified by having a fire burning on it,
or by being sprayed by a firefighter.  Both methods reduce the flammability
of the space.  If the flammability of a space is zero, there is no chance
for a fire to start there.  If a fire is extinguished on a space before the
flammability reaches zero, there is still opportunity for a new fire to
burn in that space.

\section{Hypotheses}

\begin{description}

  \item \textbf{Hypothesis 1:}
    The effectiveness of the team of firefighter agents is proportional to
    $\frac{N}{K}$.  However, if $N \ll K$, then the effectiveness becomes
    non-existent, and the fire will end up burning itself out, potentially
    killing all of the firefighters.
  
  \item \textbf{Hypothesis 2:}
    The effectiveness of the team is directly proportional to $E$, but
    inversely proportional to $P$.  That is, if a firefighter has more 
    energy, it is more efficient.  However, the more they are affected by
    being adjacent to a firefighter that dies, the less efficient they will
    be.

  \item \textbf{Hypothesis 3:}
    If $L \times L \gg N$ and $K \gg N$, then the system will not achieve
    coherence.  On the other hand, if $N \gg L \times L$ and $N \gg K$, then
    the system will achieve coherence quickly.

  \item \textbf{Hypothesis 4:}
    The effectiveness of the team is inversely proportional to the $\frac{F}{L}$.
    That is, the higher the concentration of flammability in the environment,
    the faster the fire will spread, and the less efficient the team will be.

\end{description}

\section{Experiments}

Our experiments are designed to test the above hypotheses.  We will run numerous
simulations varying the following key parameters:

\bigskip
\begin{tabular}{ | l | l | }
  \hline
  {\bf Parameter} & {\bf Range of Values} \\
  \hline
  $L$ & 100, 250, 500, 750, 1000\\
  \hline
  $N$ & 100, 500, 1000 \\
  \hline
  $K$ & 100, 500, 1000 \\
  \hline
  $P$ & 1\%, 10\%, 25\%, 50\% \\
  \hline
  $E$ & 100, 500, 1000 \\
  \hline
  $F$ & 100, 1000, 10000 \\
  \hline
\end{tabular}
\bigskip

Thus, there are $5 \times 3 \times 3 \times 4 \times 3 \times 3 = 1620$ different
configurations.  For each configuration we will run five simulations and
average the results.  For each run, we will collect the number of firefighters
that survive, the total amount of energy of the surviving firefighters, 
how many fires were extinguished, and how many fires burned themselves out.
The first two values will determine the efficiency of the system, while
the last two will determine the effectiveness.  We will then plot the results
accordingly to respond to the hypotheses posed.

If we are able to achieve coherence, we would also like to see how different
flammability patterns across the area affect the outcome.

\end{document}
