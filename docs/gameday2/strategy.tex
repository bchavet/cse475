\documentclass{article}

\title{CSE475 Gameday \#2 Strategy}
\author{Team Win: Ben Chavet, Justin McKinstry, Steve Mott}
\date{\today}

\begin{document}
\maketitle

Our dominant strategy is to complete as many tasks as possible by
place a large number of short-lived announcements.

\section{Announcement Strategy}

At the beginning of each round, we will calculate the total number of subtasks
that we need in order to complete all of our tasks.  We will then divide those
subtasks such that we can make two-minute announcements, staggering the start
of each announcement by one minute, for eight minutes.  If all goes in our favor,
we will have all of our tasks completed at the end of eight minutes.  Assuming
that is not going to be the case, we have another eight minutes to re-announce
any subtasks that we do not have completed.  At the end of 16 minutes, we will
assess the completeness of each task and decide to either sell those subtasks
or pursue the remaining subtasks we need.

Total number of subtasks per announcement:

\[ \frac{TotalSubtasks}{8} = \frac{Subtasks}{Announcement} \]

Number of announcements each task will take:

\[ \forall{Tasks}, \frac{Subtasks_{Task_i}}{\frac{Subtasks}{Announcement}} = \frac{Annoucements}{Task_i} \]

Number of subtasks per task in each announcement:

\[ \forall{Subtask \in Task_i}, \frac{Subtask_j}{\frac{Announcements}{Task_i}} = \frac{Subtasks_{Task_i}}{Announcement} \]

\section{Bidding Strategy}

Take any extra subtasks that will not apply to our tasks, and use them to
place bids.  Given the aggressive nature of our announcement strategy,
we will sacrifice bidding until we have either completed all of our tasks
or we reach sixteen minutes.  At that point we will start bidding our
extra subtasks.

\end{document}